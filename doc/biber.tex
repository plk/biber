\documentclass{ltxdockit}
\usepackage[british]{babel}
\usepackage[strict=true,autostyle=true]{csquotes}
\usepackage{ifthen}
\usepackage{fontspec}
\setmainfont[Ligatures=TeX]{TeXGyrePagella}
\setsansfont{Arial}
\setmonofont{Courier New}

\MakeAutoQuote{«}{»}

\titlepage{%
  title={biber},
  subtitle={A backend bibliography processor for biblatex},
  url={http://biblatex-biber.sourceforge.net},
  author={François Charette, Philip Kime},
  email={firmicus@ankabut.net, Philip@kime.org.uk},
  revision={0.5.5},
  date={\today}}

\hypersetup{%
  pdftitle={biber},
  pdfsubject={A backend bibliography processor for biblatex},
  pdfauthor={Philip Kime},
  pdfkeywords={biblatex, bibliography}}



\def\biberex#1{\hbox{\hspace{-4em}\texttt{\small \detokenize{#1}}}}

\begin{document}

\printtitlepage
\tableofcontents

\section{Introduction}
\label{int}

\subsection{About}

All about \verb+biber+

\subsection{Requirements}\label{ref:req}

\verb+biber+ is distributed in two ways. There is a perl source
version which requires you to have a working perl installation
(preferably version 5.12 but no less than 5.10) and the ability to
install the pre-requisite modules. Also provided are binaries for
major OSes built with the perl \verb+PAR::Packer+ module and utilities.

Currently there are binaries available for:

\begin{itemize}
\item OSX Intel 64-bit
\item Windows
\item Linux 32-bit
\item Linux 64-bit
\end{itemize}

These should work on any fairly recent OS version. Both binaries and
perl source are available on SourceForge (\url{http://sourceforge.net/projects/biblatex-biber/}).

\subsection{License}

\verb+biber+ is released under the free sortware Artistic License 2.0\footnote{\url{http://www.opensource.org/licenses/artistic-license-2.0.php}}

\subsection{History}

\verb+bibtex+ has been the default (only \ldots) integrated choice for
bibliography processing in TeX for a long time. It has well known
limitations which stem from its data format, data model and lack of
UTF-8 support\footnote{In fact, there is now a UTF-8 compliant version
of \verb+bibtex+}. The \verb+.bst+ language for writing bibliography
styles is painful to learn and use. It is not a general programming
language and this make it really very hard to do sophisticated
automated processing of bibliographies.

\verb+biblatex+ was a major advance for LaTeX users as it moved much
of the bibliography processing into LaTeX macros. However,
\verb+biblatex+ still used \verb+bibtex+ as a sorting engine for the
bibliography and also to generate various labels for
entries. \verb+bibtex+'s capabilities even for this reduced set of
tasks was still quite restricted due to the lack of UTF-8 support and
the more and more complex programming issues involved in label
preparation and file encoding.

\verb+biber+ was designed specifically for \verb+biblatex+ in order to
provide a powerful backend engine which could deal with any required
tasks to do with \verb+.bbl+ preparation. It can

\begin{itemize}
\item Deal with the full range of UTF-8
\item Sort in a completely customisable manner, using when available
  CLDR collation tailorings
\item Automatically encode the \verb+.bbl+ into any support encoding
  format
\item Process all bibliography sections in one pass of the tool
\item Handle UTF-8 citekeys and filenames (given a suitable fully
  UTF-8 compliant TeX engine)
\item Handle very complex auto-expansion and contraction of names and
  namelists.
\item Lots of other things.
\end{itemize}

\subsection{Acknowledgments}

François Charette originally wrote \verb+biber+. Philip Kime joined in
the development in 2009.

\section{Use}\label{ref:use}
\label{use}

.bcf interface etc. conf file. command-line precedence

\subsection{Collation and Localisation}

\verb+biber+ takes care of collating (sorting) the bibliography for
\verb+biblatex+. It writes entries to the \verb+.bbl+ file sorted by a
completely customisable set of rules which are passed in the
\verb+.bcf+ file by \verb+biblatex+. \verb+biber+ has two ways of performing
collation\\[2ex]

\biberex{--collate|-C} The default. This option makes \verb+biber+ use the
  \verb+Unicode::Collate+ module for collation which implements the full UCA (Unicode
  Collation Algorithm). It also has CLDR (Common Locale Data
  Repository) tailoring to deal with cases which are not covered by the
  UCA. It is a little slower than \verb+--fastsort|-f+ but the
  advantages are such that it's rarely worth using \verb+--fastsort|-f+\\[1ex]

\biberex{--fastsort|-f} Biber will sort using
  the OS locale collation tables. The drawback for this method is that special
  collation tailoring for various languages are not implemented in the
  collation tables for many OSes. For example, few OSes correctly sort 'å'
  before 'ä' in the Swedish (\verb+sv_SE+) locale. If you are using a
  common latin alphabet, then this is probably not a problem for you.\\[2ex]

\noindent The locale used for sorting is determined by the following resource
chain in decreasing precedence order:\\[2ex]

\noindent\verb+--collate_options|-c+ $\rightarrow$\\
\hspace*{1em}\verb+--sortlocale|-c+ $\rightarrow$\\
\hspace*{2em}\verb+LC_COLLATE+ environment variable $\rightarrow$\\
\hspace*{3em}\verb+LANG+ environment variable $\rightarrow$\\
\hspace*{4em}\verb+LC_ALL+ environment variable\\[2ex]

\noindent With the default \verb+--collate|-C+ option, the locale will
be used to look for a collation tailoring for that locale. It will generate an
information warning if it finds none. This is not a problem as the vast
majority of collation cases are covered by the basic standard UCA and many
locales neither have nor need any special collation tailoring.

\noindent With the \verb+--fastsort|-f+ option, the locale will be
used to locate an OS locale definition to use for the collation. This
may or may not be correctly tailored, depending on the locale and the OS.

\noindent Collation is by default case sensitive. You can turn this
off using the \verb+biber+ option \verb+--sortcase=0+ or set the
\verb+biblatex+ option \verb+sortcase=false+.

\noindent \verb+--collate|-C+ by default collates uppercase before
lower. You can reverse this using the \verb+biber+ option \verb+--sortupper=0+
or set the \verb+biblatex+ option \verb+sortupper=false+.

\noindent There are in fact many options to \verb+Unicode::Collate+
which can tailor the collation in various ways in
addition to the locale tailoring which is automatically performed.
Users should see the the documentation to the module for the various
options, most of which the vast majority of users will never
need\footnote{For details on the various options, see
  \url{http://search.cpan.org/search?query=Unicode%3A%3ACollate&mode=all}}.
Options are passed as a single quoted string, each option separated by
comma, each option and value separated by \verb+=>+. See examples.

\subsubsection{Examples}

\biberex{\verb+biber+}

\noindent Call biber using all settings from the \verb+.bcf+ generated from the
LaTeX run. Case sensitive UCA sorting is performed taking the locale
for tailoring from the environment

\biberex{\verb+biber --sortlocale=de_DE+}

\noindent Override the locale setting possibly in the \verb+.bcf+ or
in the environment.

\biberex{\verb+biber --fastsort+}

\noindent Use slightly quicker internal sorting routine. This uses the OS locale
files which may or may not be accurate.

\biberex{\verb+biber --sortcase=0+}

\noindent Case insensitive sorting.

\biberex{\verb+biber --sortupper=0 --collate_options="backwards => 2"+}

\noindent Collate lowercase before upper and collate French accents in
reverse order at UCA level 2.

\subsection{Encoding of \verb+.bib+ and \verb+.bbl+ files}

\verb+biber+ takes care of reencoding the \verb+.bib+ data as
necessary. In normal use, \verb+biblatex+ passes its
\verb+bibencoding+ option value to \verb+biber+ via the \verb+.bcf+
file. It also passes an option \verb+bblencoding+ the value of which
is derived from the \verb+inputenc+ package setting (if the user is
using this), otherwise «utf8» (for XeTeX or LuaTeX) or «ascii» (any
other TeX engine).

\noindent \verb+biber+ performs the following tasks:

\begin{enumerate}
\item Decodes the \verb+.bib+ into UTF-8 if it is not UTF-8 already
\item Decodes LaTeX character macros into UTF-8 unless \verb+--latexdecode=0+
\item Encodes the output so that the \verb+.bbl+ is in
  the encoding that \verb+bblencoding+ specifies.
\item Warns if it is asked to output to the \verb+.bbl+ any UTF-8
  decoded LaTeX character macros which are not in the
  \verb+bblencoding+ encoding. Replaces with a diacritic-stripped substitute.
\end{enumerate}

\noindent As you can see from item 2 above, by default, \verb+biber+
converts LaTeX character macros into UTF-8 internally. This is very
useful as it means that things are sorted correctly but has two
potential (but rare) problems which you should be aware of:

\begin{itemize}
\item If you are using LaTeX and \verb+\usepackage[utf8]{inputenc}+, it is possible
  that the UTF-8 characters resulting from \verb+biber+'s
  internal LaTeX character macro decoding break \verb+inputenc+. This
  is because \verb+inputenc+ does not implement all of UTF-8, only a
  commonly used subset. The solution is to either use the \verb+biber+
  option \verb+--latexdecode=0+ in which case the sorting might not be
  nice because then all \verb+biber+ can do is strip away the LaTeX
  markup in order to internally generate sorting data.

  So if you had perhaps \verb+\DJ+ in your \verb+.bib+,
  \verb+biber+ decodes this correctly to «Đ» and this breaks \verb+inputenc+
  because it doesn't understand that UTF-8 character. You could set
  \verb+--latexdecode=0+ but then \verb+biber+ would strip this
  internally for generating sorting data to «» which is a different
  character entirely and sorts in a completely different place.

  The best solution here is to switch to a TeX engine with full
  UTF-8 suport like XeTeX or LuaTeX as these don't use or need
  \verb+inputenc+.
\item If your \verb+bblencoding+ is not UTF-8, and you are using some
  UTF-8 equivalent LaTeX character macros in your \verb+.bib+, then
  some \verb+.bbl+ fields (currently only \verb+\sortinit{}+) might
  end up with invalid character in them, according to the \verb+.bbl+
  encoding. This is because some fields must be generated from the
  final sorting data which is only available after the LaTeX character
  macro decoding step.

  For example, suppose you were using PDFLaTeX with
  \verb+\usepackage[latin1]{inputenc}+ and the following
  \verb+.bib entry+

  \begin{verbatim}
    @BOOK{citekey1,
      AUTHOR = {{\v S}imple, Simon},
    }
  \end{verbatim}

  \noindent With normal LaTeX character macro decoding, the
  \verb+{\v S}+ is decoded into «Š» and so with name-first sorting,
  \verb+\sortinit{}+ would be «Š». This is an invalid character in
  latin1 encoding and so the \verb+.bbl+ would be broken. In such
  cases when \verb+\sortinit{}+ is a char not valid in the
  \verb+bblencoding+, \verb+biber+ strips off any diacritics which in
  this case results in «S». This is not ideal as this is not the
  initial character of the string used for sorting any more but it's a
  decent replacement in such cases. You could also turn off decoding
  of LaTeX character macros with \verb+--latexdecod=0+ but then the
  comments about sorting in the case above apply.

  Really, you should use UTF-8 \verb+bblencoding+ wherever possible
  and again, this might mean switching TeX engines to one that
  supports full UTF-8.
\end{itemize}

\noindent Normally, you do not need to set the encoding options on the
\verb+biber+ command line as they are passed in the \verb+.bcf+ via
the information in your \verb+biblatex+ environment. However, you can
override the \verb+.bcf+ settings with the command line.

\subsubsection{Examples}

\biberex{\verb+biber+}

\noindent Read \verb+bibencoding+ and \verb+bblencoding+ from the
\verb+.bcf+.

\biberex{\verb+biber --bblencoding=latin2+}

\noindent Encode the \verb+.bbl+ as latin2, overriding the
\verb+.bcf+.

\biberex{\verb+biber --bblencoding=latin1 --latexdecode=0+}

\noindent Encode the \verb+.bbl+ as latin1, overriding the
\verb+.bcf+. Don't decode any LaTeX character macros into UTF-8 internally.

\biberex{\verb+biber -u+}

\noindent Shortcut alias for \verb+biber --bibencoding=UTF-8+

\biberex{\verb+biber -U+}

\noindent Shortcut alias for \verb+biber --bblencoding=UTF-8+

\subsection{Limitations}
\label{use:limit}

Custom entry types/fields. List uniqueness etc.


\end{document}
