% To make this maximally compatible with older PDF readers, it uses
% PDF 1.4 and TeXGyre fonts
\pdfminorversion=4
\documentclass{ltxdockit}
\usepackage[british]{babel}
\usepackage[strict=true,autostyle=true]{csquotes}
\usepackage{ifthen}
\usepackage{fontspec}
\usepackage{graphicx}
\usepackage{booktabs}
\usepackage{fixfoot}
\usepackage{color}
\usepackage{listings}
\setmainfont[Ligatures=TeX]{TeXGyrePagella}
\setsansfont[Ligatures=TeX]{TexGyreHeros}
\setmonofont[Ligatures=NoCommon]{TeXGyreCursor}

\MakeAutoQuote{«}{»}

\DeclareFixedFootnote{\tpb}{Binary maintained by third party. See README in
  binary download directory for this platform for support/contact
  details. Usually, the binary maintainer is also the binary build
  provider for TexLive.}

\titlepage{%
  title={biber},
  subtitle={A backend bibliography processor for biblatex},
  url={http://biblatex-biber.sourceforge.net},
  author={François Charette, Philip Kime},
  email={firmicus@ankabut.net, Philip@kime.org.uk},
  revision={biber 0.9.5 (biblatex 1.6)},
  date={\today}}

\hypersetup{%
  pdftitle={biber},
  pdfsubject={A backend bibliography processor for biblatex},
  pdfauthor={Philip Kime},
  pdfkeywords={biblatex, bibliography}}


% Control list spacing
\usepackage{enumitem}
\setdescription{noitemsep}
\setenumerate{noitemsep}
\setitemize{noitemsep}

\def\biberex#1{\hbox{\hspace{-4em}\texttt{\small \detokenize{#1}}}}

\begin{document}
\definecolor{grey}{rgb}{0.7,0.7,0.7}
\printtitlepage
\tableofcontents

\section{Introduction}
\label{int}

\subsection{About}

\verb+biber+ is conceptually a \verb+bibtex+ replacement for
\verb+biblatex+. It is written in \verb+Perl+ with the aim of providing a
customised and sophisticated data preparation backend for \verb+biblatex+.
You do \emph{not} need to install \verb+Perl+ use \verb+biber+---binaries
are provided for many operating systems via the main TeX
distributions (TeXLive, MacTeX, MiKTeX) and also via download from SourceForge.
Functionally, \verb+biber+ offers a superset of \verb+bibtex+'s capabilities but is
tightly coupled with \verb+biblatex+ and cannot be used as a stand-alone tool
with standard \verb+.bst+ styles. \verb+biber+'s role is to support
\verb+biblatex+ by performing the following tasks:

\begin{itemize}
\item Parsing data from data sources
\item Processing cross-references, entry sets, related entries
\item Generating data for name, name list and name/year disambiguation
\item Structural validation according to \verb+biblatex+ data model
\item Sorting reference lists
\item Outputting data to a \verb+.bbl+ for \verb+biblatex+ to consume
\end{itemize}

\subsection{Requirements}\label{ref:req}

\verb+biber+ is distributed primarily as a stand-alone binary and is
included in TeXLive, MacTeX and MiKTeX. If you are using any of these
distributions, you do not need any additional software installed to use
\verb+biber+. You do \emph{not} need a \verb+Perl+ installation at all to use
the binary distribution of \verb+biber+\footnote{If you prefer, you can run
\texttt{biber} as a normal \texttt{Perl} program and doing this \emph{does} require
you to have a \texttt{Perl} interpreter installed. See section \ref{binaries}.}.

\verb+biber+ is developed on
SourceForge\footnote{\url{http://sourceforge.net/projects/biblatex-biber/}}
and this is the primary location for development releases, forums and
bugfixes etc. It is included into TeXLive from the SourceForge releases.

\subsection{License}

\verb+biber+ is released under the free software Artistic License 2.0\footnote{\url{http://www.opensource.org/licenses/artistic-license-2.0.php}}

\subsection{History}

\verb+bibtex+ has been the default (only \ldots) integrated choice for
bibliography processing in TeX for a long time. It has well known
limitations which stem from its data format, data model and lack of Unicode
support\footnote{In fact, there is now a Unicode version}. The
\verb+.bst+ language for writing bibliography styles is painful to learn
and use. It is not a general programming language and this makes it really
very hard to do sophisticated automated processing of bibliographies.

\verb+biblatex+ was a major advance for LaTeX users as it moved much
of the bibliography processing into LaTeX macros. However,
\verb+biblatex+ still used \verb+bibtex+ as a sorting engine for the
bibliography and also to generate various labels for
entries. \verb+bibtex+'s capabilities even for this reduced set of
tasks was still quite restricted due to the lack of Unicode support and
the more and more complex programming issues involved in label
preparation and file encoding.

\verb+biber+ was designed specifically for \verb+biblatex+ in order to
provide a powerful backend engine which could deal with any required
tasks to do with \verb+.bbl+ preparation. Its main features are:

\begin{itemize}
\item Deals with the full range of UTF-8
\item Sorts in a completely customisable manner, using when available,
  CLDR collation tailorings
\item Allows for per-entrytype options
\item Automatically encodes the \verb+.bbl+ into any supported encoding
  format\footnote{«Supported» here means encodings supported by the
    \texttt{Perl} \texttt{Encode} module}
\item Processes all bibliography sections in one pass of the tool
\item Handles UTF-8 citekeys and filenames (given a suitable fully
  UTF-8 compliant TeX engine)
\item Creates entry sets dynamically and allow easily defined static entry sets,
  all processed in one pass
\item Flexible user-customisable crossreference field inheritance
  model
\item Handles complex auto-expansion and contraction of names and
  namelists\footnote{Coming in BibLaTeX 1.4}
\item Support for related entries, to enable generic treatment of things
  like «translated as», «reprinted as», «reprint of»
  etc.\footnote{Coming in BibLaTeX 1.5}
\item Extensible modular data sources architecture for ease of adding
  more data source types.
\item Support for remote data sources
\end{itemize}

\subsection{Performance}

\verb+biber+ can't really be compared with \verb+bibtex+ in any meaningful
way performance-wise. \verb+biber+ is written in \verb+Perl+ and does a
great deal more than \verb+bibtex+ which is written in C. One of
\verb+biber+'s test cases is a 2150 entry, 15,000 line \verb+.bib+ file
which references a 630 entry macros file with a resulting 160 or so page (A4)
formatted bibliography. This takes \verb+biber+ about 100 seconds on
average to process on a reasonable computer. This is perfectly acceptable,
especially for a batch program \ldots

\subsection{Acknowledgements}

François Charette originally wrote \verb+biber+. Philip Kime joined in
the development in 2009.

\section{Use}

Firstly, please note that \verb+biber+ will \emph{not} attempt to sanitise
the content of \verb+bibtex+ data sources. That is, don't expect it to
auto-escape any TeX special characters like «\verb+&+» or «\verb+%+» which
it finds in, for example, your \verb+TITLE+ fields. It used to do this in
earlier versions in some cases but as of version 0.9, it doesn't because
it's fraught with problems and leads to inconsistent expectations and
behaviour between different data source types. In your \verb+bibtex+ data
sources, please make sure your entries are legal TeX code.

Running \verb+biber --help+ will display all options and a brief
description of each. This is the most useful brief source of usage
information. \verb+biber+ returns an exit code of 0 on success or 1 if
there was an error.

Most \verb+biber+ options can be specified in long or short format. When
mentioning options below, they are referred to as
«\verb+long form|short form+» when an option has both a long and short
form. As usual with such options, when the option requires an argument, the
long form is followed by an equals sign «\verb+=+» and then the argument,
the short form is followed by a space and then the argument. For example,
the \verb+--configfile|-g+ option can be given in two ways:

\begin{verbatim}
biber --configfile=somefile.conf
biber -g somefile.conf
\end{verbatim}

With the \verb+backend=biber+ option, \verb+biblatex+ switches its backend
interface and passes all options and information relevant to \verb+biber+'s
operation in a control file with extension \verb+.bcf+\footnote{BibLaTeX Control
  File}. This is conceptually equivalent to the \verb+.aux+ file which
LaTeX uses to pass information to \verb+bibtex+. The \verb+.bcf+ file is
XML and contains many options and settings which configure how \verb+biber+
is to process the bibliography and generate the \verb+.bbl+ file.

The usual way to call \verb+biber+ is simply with the \verb+.bcf+ file
as the only argument. The «\verb+.bcf+» extension of the control file
is not optional. \verb+biblatex+ always outputs a control file with
the \verb+.bcf+ extension. Specifying the «\verb+.bcf+» extension to
\verb+biber+ \emph{is} optional. Assuming a control file called
\verb+test.bcf+, the following two commands are equivalent:

\begin{verbatim}
biber test.bcf
biber test
\end{verbatim}

\subsection{Options and config file}
\verb+biber+ sets its options using the following resource 
chain which is given in decreasing precedence order:\\[2ex]

\noindent command line options $\rightarrow$\\
\hspace*{1em}\verb+biber.conf+ file $\rightarrow$\\
\hspace*{2em}\verb+.bcf+ file$\rightarrow$\\
\hspace*{3em}\verb+biber+ hard-coded defaults\\[2ex]

\noindent Users do not need to care directly about the contents or format of the
\verb+.bcf+ file as this is generated from the options which they specify
for \verb+biblatex+. The config file is a place to set commonly used
command-line options and also to set options which cannot be set on the
command line.

The configuration file is by default called \verb+biber.conf+ but this can
be changed using the \verb+--configfile|-g+ option. Unless
\verb+--configfile|-g+ is used, the config file is
looked for in the following places, in decreasing order of preference:\\[2ex]

\noindent \verb+biber.conf+ in the current directory $\rightarrow$\\
\hspace*{1em}\verb+$HOME/.biber.conf+ $\rightarrow$\\
\hspace*{2em}\verb+$XDG_CONFIG_HOME/biber/biber.conf+ $\rightarrow$\\
\hspace*{3em}\verb+$HOME/Library/biber/biber.conf+ (Mac OSX only)\\
\hspace*{3em}\verb+$APPDATA/biber.conf+ (Windows only) $\rightarrow$\\
\hspace*{4em}the output of «\verb+kpsewhich biber.conf+» (if available on the
system)\\[2ex]

\noindent The config file format is a very flexible one which allows users to specify
options in most common formats, even mixed in the same file. It's easier to
see an example. Here is a config file which displays the \verb+biber+
defaults:

\begin{verbatim}
bblencoding         UTF-8
bibencoding         UTF-8
collate             1
<collate_options>
    level           3
</collate_options>
debug               0
fastsort            0
mincrossrefs        2
nolog               0
nostdmacros         0
<nosort>
  # ignore prefices like 'al-' when sorting name fields
  type_names      \A\p{L}{2}\p{Pd}
  # ignore diacritics when sorting author
  type_names      [\x{2bf}\x{2018}]
</nosort>
onlylog             0
quiet               0
sortcase            true
sortlocale          en_US.utf8
sortupper           true
trace               0
validate_control    0
validate_structure  0
wraplines           0
\end{verbatim}

\noindent You can see here that options with multiple key/value pairs of
their own like\linebreak[4] \verb+--collate_options|-c+ can be specified in
Apache config format. Please see the documentation
for the \verb+Config::General+ \verb+Perl+
module\footnote{\url{http://search.cpan.org/search?query=Config::General&mode=all}}
if you really need details. In practise, if you use a config file at all
for \verb+biber+, it will contain very little as you will usually set all
options by setting options in \verb+biblatex+ which will pass them to
\verb+biber+ via the \verb+.bcf+ file.

The \verb+--collate_options|-c+ option takes a number of key/value pairs as
value. See section \ref{coll} for details.

\subsubsection{The «map» option}

The supplied data source drivers implement a default mapping from
data source entrytypes and fields into the \verb+biblatex+ data model. If
you want to override or augment the driver mappings you can use the
\verb+map+ option which makes it possible to, for example, have a data
source with non-standard entrytypes or fields and to have these
automatically mapped into valid \verb+biblatex+ data types without
modifying your data source.

The \verb+map+ option can only be set in the config file
and not on the command line as it has a complex structure. This
option allows you to perform various data source mapping (aliasing)
tasks:

\begin{itemize}
\item Map data source entrytypes to \verb+biblatex+ entrytypes,
  optionally also adding new fields to the entry.
\item Map data source fields to \verb+biblatex+ fields,
  optionally also adding new fields to the entry. This is basically a
  one$\rightarrow$many field mapping. You can also limit field
  mappings to specific data source entrytypes.
\item As a special case of the above, you can map data source fields
  to null, effectively removing them from the input stream.
\end{itemize}

\noindent The format of the \verb+map+ option in the
config file is described below. Items in \textcolor{red}{red} are not
literal, they are descriptive meta-values which are explained in the
accompanying text. Items in \textcolor{blue}{blue} are optional within
their parent section. The entire \verb+map+ option is not case sensitive
\emph{apart from the angle-bracketed section names which must be
  lower-cased as shown}. The examples below use uppercase for variable names
and setting in places, for clarity. The general structure is:

\lstset{showspaces=false}
\begin{lstlisting}[escapechar=+,mathescape=true]
<map>
  <+\textcolor{red}{driver1}+>
    +\textcolor{blue}{BMAP\_OVERWRITE 1|0}+
       ...
  </+\textcolor{red}{driver1}+>
    +\textcolor{blue}{BMAP\_OVERWRITE 1|0}+
       $\vdots$
  <+\textcolor{red}{drivern}+>
    +\textcolor{blue}{BMAP\_OVERWRITE 1|0}+
       ...
  </+\textcolor{red}{drivern}+>
</map>
\end{lstlisting}

\noindent Here, \textcolor{red}{\texttt{driver1}}\ldots
\textcolor{red}{\texttt{drivern}} are the names of valid \verb+biber+ data
source drivers (see section \ref{dcf}). Reserved tokens with special meanings for
this option have the prefix «\verb+BMAP_+» and are described below. Each
driver section contains one or more of the following subsections:

\begin{itemize}
\item Entrytype mapping
\item Simple global field mapping
\item Extended global field mapping
\item Entrytype specific field mapping
\end{itemize}

\minisec{Entrytype mapping}

\lstset{showspaces=false}
\begin{lstlisting}[escapechar=+,mathescape=true]
<entrytype +\textcolor{red}{source-entrytype}+>
  BMAP_TARGET +\textcolor{red}{target-entrytype}+
  +\textcolor{blue}{<alsoset>}+
    +\textcolor{red}{target-field1 target-value1}+
       $\vdots$
    +\textcolor{red}{target-fieldn target-valuen}+
  +\textcolor{blue}{</alsoset>}+
</entrytype>
\end{lstlisting}

\noindent An \verb+entrytype+ specification maps the data source
entrytype \textcolor{red}{\texttt{source-entrytype}} into the \verb+biblatex+
entrytype \textcolor{red}{\texttt{target-entrytype}}. The literal string \verb+BMAP_TARGET+
is a constant and must be used to specify the target entrytype. The
\textcolor{blue}{\texttt{alsoset}} section is optional and specifies one
or more \textcolor{red}{\texttt{target-field}}s to set in the entry, along
with their \textcolor{red}{\texttt{target-value}}s. The literal token
\verb+BMAP_ORIGENTRYTYPE+ can be used as the
\textcolor{red}{\texttt{target-value}} and it resolves to the data source
entrytype name. If the option \textcolor{blue}{\texttt{BMAP\_OVERWRITE}} is
true (set to «1»), then any existing fields which the
\textcolor{blue}{\texttt{alsoset}} section mentions will be overwritten. If
this option is missing or false (set to «0»), then the field will not be
set and the existing field value will remain. In both cases, warnings will
be issued about the conflicting field, saying whether it will be
overwritten or not. An example:

\lstset{showspaces=false}
\begin{lstlisting}[escapechar=+,mathescape=true]
<map>
  <bibtex>
    BMAP_OVERWRITE 1
    <entrytype CONVERSATION>
      BMAP_TARGET CUSTOMA
      <alsoset>
        USERA  BMAP_ORIGENTRYTYPE
        NOTE "Auto-created this field"
      </alsoset>
    </entrytype>
    <entrytype RECORD>
      BMAP_TARGET MUSIC
    </entrytype>
  </bibtex>
</map>
\end{lstlisting}

\noindent This would turn a \verb+CONVERSATION+ entrytype in all
\verb+bibtex+ data sources into a \verb+CUSTOMA+ entrytype
and would automatically set the \verb+USERA+ field in the entry to
«conversation» and the \verb+NOTE+ field to «Auto-created this field». The
\verb+USERA+ and \verb+NOTE+ fields will be overwritten if present in the
entry. Also, the example turns a \verb+RECORD+ entrytype into a \verb+MUSIC+ entrytype.
\bigskip
\minisec{Simple global field mapping}

\lstset{showspaces=false}
\begin{lstlisting}[escapechar=+,mathescape=true]
<globalfield>
  +\textcolor{red}{source-field1 target-field1}+
                 $\vdots$
  +\textcolor{red}{source-fieldn target-fieldn}+
</globalfield>
\end{lstlisting}

\noindent This section maps data source fields to \verb+biblatex+ fields,
regardless of the entrytypes in which they occur (hence «global»). You may
use the special \textcolor{red}{\texttt{target-field}} constant \verb+BMAP_NULL+
to map a field to null, that is, to ignore it. This essentially has the
effect of treating the field as if it did not exist in the data source. For
example (to add to the previous example, so that you can see a more
complete mapping example. The previously discussed example settings are greyed):

{\color{grey}
\lstset{showspaces=false}
\begin{lstlisting}[escapechar=+,mathescape=true]
<map>
  <bibtex>
    BMAP_OVERWRITE 1
    <entrytype CONVERSATION>
      BMAP_TARGET CUSTOMA
      <alsoset>
        USERA  BMAP_ORIGENTRYTYPE
        NOTE "Auto-created this field"
      </alsoset>
    </entrytype>
    <entrytype RECORD>
      BMAP_TARGET MUSIC
    </entrytype>
    +\textcolor{black}{<globalfield>}+
      +\textcolor{black}{ABSTRACT BMAP\_NULL}+
      +\textcolor{black}{PARTICIPANT NAMEA}+
    +\textcolor{black}{</globalfield>}+
  </bibtex>
  +\textcolor{black}{<ris>}+
    +\textcolor{black}{N2 BMAP\_NULL}+
  +\textcolor{black}{</ris>}+
</map>
\end{lstlisting}
}

\noindent Here, we have specified that we should ignore (map to
\verb+BMAP_NULL+) any \verb+ABSTRACT+ fields and that we should map all
\verb+PARTICIPANT+ fields to the \verb+NAMEA+ field. We have also specified
that we should ignore the \verb+N2+ field in \verb+ris+ data sources (this
is often used for abstracts in RIS).
\bigskip
\minisec{Extended Global field mapping}

\lstset{showspaces=false}
\begin{lstlisting}[escapechar=+,mathescape=true]
<globalfield +\textcolor{red}{source-field}+>
  BMAP_TARGET +\textcolor{red}{target-field}+
  +\textcolor{blue}{<alsoset>}+
    +\textcolor{red}{target-field1 target-value1}+
               $\vdots$
    +\textcolor{red}{target-fieldn target-valuen}+
  +\textcolor{blue}{</alsoset>}+
</globalfield>
\end{lstlisting}

\noindent Unlike the simple global mapping, this form maps only one
\textcolor{red}{\texttt{source-field}}. You must specify the
\textcolor{red}{\texttt{target-field}} using the constant
\verb+BMAP_TARGET+. The \textcolor{blue}{\texttt{alsoset}} section is
optional and specifies one or more \textcolor{red}{\texttt{target-field}}s
to set in the entry, along with their
\textcolor{red}{\texttt{target-value}}s. The literal token
\verb+BMAP_ORIGFIELD+ can be used as the
\textcolor{red}{\texttt{target-value}} and it resolves to the data source
field name. Continuing our example configuration:

{\color{grey}
\lstset{showspaces=false}
\begin{lstlisting}[escapechar=+,mathescape=true]
<map>
  <bibtex>
    BMAP_OVERWRITE 1
    <entrytype CONVERSATION>
      BMAP_TARGET CUSTOMA
      <alsoset>
        USERA  BMAP_ORIGENTRYTYPE
        NOTE "Auto-created this field"
      </alsoset>
    </entrytype>
    <entrytype RECORD>
      BMAP_TARGET MUSIC
    </entrytype>
    <globalfield>
      ABSTRACT BMAP_NULL
      PARTICIPANT NAMEA
    </globalfield>
    +\textcolor{black}{<globalfield PUBMEDID>}+
       +\textcolor{black}{BMAP\_TARGET  EPRINT}+
      +\textcolor{black}{<alsoset>}+
        +\textcolor{black}{EPRINTTYPE BMAP\_ORIGFIELD}+
        +\textcolor{black}{NOTE "Auto-created this field"}+
      +\textcolor{black}{</alsoset>}+
    +\textcolor{black}{</globalfield>}+
  </bibtex>
  <ris>
    N2 BMAP_NULL
  </ris>
</map>
\end{lstlisting}
}

\noindent Here, in all entries, the \verb+PUBMEDID+ field becomes the
\verb+EPRINT+ field and the \verb+EPRINTTYPE+ field is set to «pubmedid».
The \verb+NOTE+ field is set to «Auto-created this field».
\bigskip
\minisec{Entrytype specific field mapping}

\lstset{showspaces=false}
\begin{lstlisting}[escapechar=+,mathescape=true]
<field +\textcolor{red}{source-field}+>
  BMAP_PERTYPE +\textcolor{red}{source-entrytype1}+
                 $\vdots$
  BMAP_PERTYPE +\textcolor{red}{source-entrytypen}+
  BMAP_TARGET +\textcolor{red}{target-field}+
  +\textcolor{blue}{<alsoset>}+
    +\textcolor{red}{target-field1 target-value1}+
                 $\vdots$
    +\textcolor{red}{target-fieldn target-valuen}+
  +\textcolor{blue}{</alsoset>}+
</field>
\end{lstlisting}

\noindent Here we map the data source
\textcolor{red}{\texttt{source-field}} to the
\textcolor{red}{\texttt{target-field}} but only if it occurs within one of
the data source entrytypes \textcolor{red}{\texttt{source-entrytype1}}
\ldots\linebreak\textcolor{red}{\texttt{source-entrytypen}}. The source
entrytype restrictions must be specified using the \verb+BMAP_PERTYPE+
token. The target field must be specified using the \verb+BMAP_TARGET+
token. The \textcolor{blue}{\texttt{alsoset}} section is optional and
specifies one or more \textcolor{red}{\texttt{target-field}}s to set in the
entry, along with their \textcolor{red}{\texttt{target-value}}s. The
literal token \verb+BMAP_ORIGFIELD+ can be used as the
\textcolor{red}{\texttt{target-value}} and it resolves to the data source
field name. If there is a global field mapping and also an entrytype
specific mapping which both apply to the same field, the more specific has
precedence. This is so you can, for example, choose to map a field globally
in one way but deal with it another way in certain entrytypes. Continuing
the example: 

{\color{grey}
\lstset{showspaces=false}
\begin{lstlisting}[escapechar=+,mathescape=true]
<map>
  <bibtex>
    BMAP_OVERWRITE 1
    <entrytype CONVERSATION>
      BMAP_TARGET CUSTOMA
      <alsoset>
        USERA  BMAP_ORIGENTRYTYPE
        NOTE "Auto-created this field"
      </alsoset>
    </entrytype>
    <entrytype RECORD>
      BMAP_TARGET MUSIC
    </entrytype>
    <globalfield>
      ABSTRACT BMAP_NULL
      PARTICIPANT NAMEA
    </globalfield>
    <globalfield PUBMEDID>
       BMAP_TARGET  EPRINT
      <alsoset>
        EPRINTTYPE BMAP_ORIGFIELD
        NOTE "Auto-created this field"
      </alsoset>
    </globalfield>
    +\textcolor{black}{<field NOTE>}+
      +\textcolor{black}{BMAP\_PERTYPE ARTICLE}+
      +\textcolor{black}{BMAP\_PERTYPE BOOK}+
      +\textcolor{black}{BMAP\_TARGET BMAP\_NULL}+
      +\textcolor{black}{<alsoset>}+
        +\textcolor{black}{ADDENDUM "A string"}+
      +\textcolor{black}{</alsoset>}+
    +\textcolor{black}{</field>}+
  </bibtex>
  <ris>
    N2 BMAP_NULL
  </ris>
</map>
\end{lstlisting}
}

\noindent Here we have specified that within data
source entrytypes \verb+ARTICLE+ and \verb+BOOK+, the field \verb+NOTE+
should be ignored and the \verb+ADDENDUM+ field should be set to «A
string».

Please note that the data source entrytypes and fields are \emph{not}
necessarily the same as in the \verb+biblatex+ data model. The definitive
guide to the \verb+biblatex+ data model is the \verb+biblatex+
documentation. It is certainly the case that for \verb+bibtex+ data
sources, there is a large degree of congruence between the data source
entrytypes and fields and the \verb+biblatex+ data model because
\verb+biblatex+ grew out of \verb+bibtex+. However, for other data sources
like \verb+ris+, \verb+endnotexml+ and \verb+zoterordfxml+, the source
entrytypes and fields are usually very differently modelled and named. For
example, here is how to drop the «subject» fields from various entrytypes
in Zotero XML RDF format data sources:

\lstset{showspaces=false}
\begin{lstlisting}[escapechar=+,mathescape=true]
<map>
  <zoterordfxml>
    <field "dc:subject">
      BMAP_PERTYPE journalArticle
      BMAP_PERTYPE book
      BMAP_PERTYPE bookSection
      BMAP_TARGET BMAP_NULL
    </field>
  </zoterordfxml>
</map>
\end{lstlisting}

\noindent Or here, mapping journal articles into \verb+REPORT+ entries for
Endnote XML format data sources.

\lstset{showspaces=false}
\begin{lstlisting}[escapechar=+,mathescape=true]
<map>
  <endnotexml>
    <entrytype "journal Article">
      BMAP_TARGET REPORT
    </entrytype>
  </endnotexml>
</map>
\end{lstlisting}
\bigskip
\subsubsection{The «nosort» option}

The value of the \verb+nosort+ option can only be set in the config file
and not on the command line. This is because the values are \verb+Perl+ regular
expressions and would need special quoting to set on the command line. This
can get a bit tricky on some OSes (like Windows) so it's safer to set them
in the config file. In any case, it's unlikely you would want to set them
for particular \verb+biber+ runs; they would more likely be set as your
personal default and thus they would naturally be set in the config file
anyway. \verb+nosort+ allows you to ignore parts of a field for sorting.
This is done using \verb+Perl+ regular expressions which specify what to
ignore in a field. You can specify as many patterns as you like for a
specific field. Also available are some field type aliases so you can, for
example, specify patterns for all name fields or all title fields. These
field types all begin with the string «\verb+type_+», see Table
\ref{tab:nst}.

\begin{table}
\begin{center}
\small
\begin{tabular}{lllll}
\toprule
Alias & Fields\\
\midrule
type\_name & author\\
          & afterword\\
          & annotator\\
          & bookauthor\\
          & commentator\\
          & editor\\
          & editora\\
          & editorb\\
          & editorc\\
          & foreword\\
          & holder\\
          & introduction\\
          & namea\\
          & nameb\\
          & namec\\
          & shortauthor\\
          & shorteditor\\
          & translator\\
type\_title & booktitle\\
           & eventtitle\\
           & issuetitle\\
           & journaltitle\\
           & maintitle\\
           & origtitle\\
           & title\\
\bottomrule
\end{tabular}
\end{center}
\caption{nosort option field type aliases}
\label{tab:nst}
\end{table}

For example, this option can be used to ignore diacritic marks and prefices
in names which should not be considered when sorting. Given (the default):

\begin{verbatim}
<nosort>
  type_names      \A\p{L}{2}\p{Pd}
  type_names      [\x{2bf}\x{2018}]
</nosort>
\end{verbatim}

and the \verb+.bib+ data source entry:

\begin{verbatim}
author	   = {{al-Hasan}, ʿAlī},
\end{verbatim}

\noindent the prefix «al-» and the diacritic «ʿ» will not be considered
when sorting. See the \verb+Perl+ regular expression manual page for
details of the regular expression syntax\footnote{\url{http://perldoc.perl.org/perlre.html}}.

If a \verb+nosort+ option is found for a specific field, it will override
any option for a type which also covers that field.

Here is another example. Suppose you wanted to ignore «The» at the
beginning of a \verb+TITLE+ field when sorting, you could add this to your
\verb+biber.conf+:

\begin{verbatim}
<nosort>
  title      \AThe\s+
</nosort>
\end{verbatim}

\noindent If you wanted to do this for all title fields listed in Table
\ref{tab:nst}, then you would do this:

\begin{verbatim}
<nosort>
  type_title      \AThe\s+
</nosort>
\end{verbatim}

\noindent \textbf{Note:} \verb+nosort+ can be specified for most fields but
not for things like dates and special fields as that wouldn't make much sense.

\subsection{Input/Output File Locations}

\subsubsection{Control file}\label{loc:cf}

The control file is normally passed as the only argument to biber. It is
searched for in the following locations, in decreasing order of
priority:\\[2ex]

\noindent Absolute filename $\rightarrow$\\
\hspace*{1em}In the \verb+--output_directory+, if specified$\rightarrow$\\
\hspace*{2em}Relative to current directory$\rightarrow$\\
\hspace*{3em}Using \verb+kpsewhich+, if available

\subsubsection{Data sources}

Local data sources of type «file» are searched for in the following
locations, in decreasing order of priority:\\[2ex]

\noindent Absolute filename $\rightarrow$\\
\hspace*{1em}In the \verb+--output_directory+, if specified$\rightarrow$\\
\hspace*{2em}Relative to current directory$\rightarrow$\\
\hspace*{3em}In the same directory as the control file$\rightarrow$\\
\hspace*{4em}Using \verb+kpsewhich+ for supported formats, if available\\[2ex]

\noindent Remote file data sources (beginning with \verb+http://+ or
\verb+ftp://+) are retrieved to a temp file and processed as normal. Users
do not specify explicitly the bibliography database files; they are passed
in the \verb+.bcf+ control file, which is constructed from the
\verb+biblatex+ \verb+\addbibresource{}+ macros.

\subsection{Logfile}

By default, the logfile for biber will be named \verb+\jobname.blg+,
so, if you run

\begin{verbatim}
  biber <options> test.bcf
\end{verbatim}

\noindent then the logfile will be called «\verb+test.blg+». Like the
\verb+.bbl+ output file, it will be created in the
\verb+--output_directory|-c+, if this option is defined. You can
override the logfile name by using the \verb+--logfile+ option:

\begin{verbatim}
  biber --logfile=lfname test.bcf
\end{verbatim}

\noindent results in a logfile called «\verb+lfname.blg+».\\

\noindent \textbf{Warning}: be careful if you are expecting \verb+biber+ to
write to directories which you don't have appropriate permissions to. This
is more commonly an issue on non-Windows OSes. For example, if you rely on
\verb+kpsewhich+ to find your database files which are in system TeX
directories, you may well not have write permission there so \verb+biber+
will not be able to write the \verb+.bbl+. Use the \verb+--outfile|-O+
option to specify the location to write the \verb+.bbl+ to in such cases.

\subsection{Collation and Localisation}\label{coll}

\verb+biber+ takes care of collating the bibliography for
\verb+biblatex+. It writes entries to the \verb+.bbl+ file sorted by a
completely customisable set of rules which are passed in the
\verb+.bcf+ file by \verb+biblatex+. \verb+biber+ has two ways of performing
collation:\\[2ex]

\biberex{--collate|-C}
  \noindent The default. This option makes \verb+biber+ use the
  \verb+Unicode::Collate+ module for collation which implements the full UCA (Unicode
  Collation Algorithm). It also has CLDR (Common Locale Data
  Repository) tailoring to deal with cases which are not covered by the
  UCA. It is a little slower than \verb+--fastsort|-f+ but the
  advantages are such that it's rarely worth using \verb+--fastsort|-f+\\[1ex]

\biberex{--fastsort|-f}
  \noindent Biber will sort using
  the OS locale collation tables. The drawback for this method is that special
  collation tailoring for various languages are not implemented in the
  collation tables for many OSes. For example, few OSes correctly sort 'å'
  before 'ä' in the Swedish (\verb+sv_SE+) locale. If you are using a
  common latin alphabet, then this is probably not a problem for you.\\[2ex]

\noindent The locale used for collation is determined by the following resource
chain which is given in decreasing precedence order:\\[2ex]

\noindent\verb+--collate_options|-c+ (e.g. \verb+-c 'locale => "de_DE"'+) $\rightarrow$\\
\hspace*{1em}\verb+--sortlocale|-l+ $\rightarrow$\\
\hspace*{2em}\verb+LC_COLLATE+ environment variable $\rightarrow$\\
\hspace*{3em}\verb+LANG+ environment variable $\rightarrow$\\
\hspace*{4em}\verb+LC_ALL+ environment variable\\[2ex]

\noindent With the default \verb+--collate|-C+ option, the locale will
be used to look for a collation tailoring for that locale. It will generate an
informational warning if it finds none. This is not a problem as the vast
majority of collation cases are covered by the standard UCA and many
locales neither have nor need any special collation tailoring.

With the \verb+--fastsort|-f+ option, the locale will be
used to locate an OS locale definition to use for the collation. This
may or may not be correctly tailored, depending on the locale and the OS.\\[1ex]

\noindent Collation is by default case sensitive. You can turn this
off globally using the \verb+biber+ option \verb+--sortcase=false+ or from
\verb+biblatex+ using its option \verb+sortcase=false+. The option can also
be defined per-field so you can sort some fields case sensitively and
others case insensitively. See the \verb+biblatex+ manual.\\[1ex]

\noindent \verb+--collate|-C+ by default collates uppercase before lower.
You can reverse this globally for all sorting using the \verb+biber+ option
\verb+--sortupper=false+ or from \verb+biblatex+ by using its option
\verb+sortupper=false+. The option can also be defined per-field so you can
sort some fields uppercase before lower and others lower before upper. See the
\verb+biblatex+ manual. Be aware though that some locales rightly enforce a
particular setting for this (for example, Danish). You will be able to
override it but \verb+biber+ will warn you if you do. \verb+sortupper+ has
no effect when using \verb+--fastsort|-f+--you are at the mercy of what
your OS locale does.\\[1ex]

There are in fact many options to \verb+Unicode::Collate+
which can tailor the collation in various ways in
addition to the locale tailoring which is automatically performed.
Users should see the the documentation to the module for the various
options, most of which the vast majority of users will never
need\footnote{For details on the various options, see
  \url{http://search.cpan.org/search?query=Unicode\%3A\%3ACollate&mode=all}}.
Options are passed using the \verb+--collate_options|-c+ option as a
single quoted string, each option separated by comma, each key and
value separated by «\verb+=>+». See examples.

\subsubsection{Examples}

\biberex{biber}

\noindent Call biber using all settings from the \verb+.bcf+ generated from the
LaTeX run. Case sensitive UCA sorting is performed taking the locale
for tailoring from the environment if no \verb+sortlocale+ is defined in
the \verb+.bcf+

\biberex{biber --sortlocale=de_DE}

\noindent Override any locale setting in the \verb+.bcf+ or
the environment.

\biberex{biber --fastsort}

\noindent Use slightly quicker internal sorting routine. This uses the OS locale
files which may or may not be accurate.

\biberex{biber --sortcase=false}

\noindent Case insensitive sorting.

\biberex{biber --sortupper=false --collate_options="backwards => 2"}

\noindent Collate lowercase before upper and collate French accents in
reverse order at UCA level 2.

\subsection{Encoding of files}

\verb+biber+ takes care of reencoding the data source data as
necessary. In normal use, \verb+biblatex+ passes its
\verb+bibencoding+ option value to \verb+biber+ via the \verb+.bcf+
file. It also passes the value of its \verb+texencoding+ option (which
maps to \verb+biber+'s \verb+bblencoding|-E+ option) the default value
of which depends on which TeX engine and encoding packages you are
using (see \verb+biblatex+ manual for details).

\noindent \verb+biber+ performs the following tasks:

\begin{enumerate}
\item Decodes the data source into UTF-8 if it is not UTF-8 already
\item Decodes LaTeX character macros into UTF-8 if \verb+--bblencoding|-E+
  is UTF-8
\item Encodes the output so that the \verb+.bbl+ is in
  the encoding that \verb+--bblencoding|-E+ specifies
\item Warns if it is asked to output to the \verb+.bbl+ any UTF-8
  decoded LaTeX character macros which are not in the
  \verb+--bblencoding|-E+ encoding. Replaces with a suitable LaTeX macro
\end{enumerate}

\noindent Normally, you do not need to set the encoding options on the
\verb+biber+ command line as they are passed in the \verb+.bcf+ via the
information in your \verb+biblatex+ environment. However, you can override
the \verb+.bcf+ settings with the command line. The resource chain for
encoding settings is, in decreasing order
of preference:\\[2ex]

\noindent\verb+--bibencoding|-e+ and \verb+--bblencoding|-E+ $\rightarrow$\\
\hspace*{1em}\verb+biber+ config file $\rightarrow$\\
\hspace*{2em}\verb+.bcf+ control file

\subsubsection{LaTeX macro decoding}

\noindent As mentioned above, \verb+biber+ sometimes converts LaTeX
character macros into UTF-8. In fact there are two situations in which
this occurs.

\begin{enumerate}
\item When \verb+--bblencoding|-E+ is UTF-8
\item Always for internal sorting purposes
\end{enumerate}

\noindent This decoding is very useful but take note of the following
two scenarios, which relate to each of the two situations in which
LaTeX macro decoding occurs:
\bigskip
\minisec{Decoding when output is UTF-8}
If you are using PDFLaTeX and \verb+\usepackage[utf8]{inputenc}+, it
is possible that the UTF-8 characters resulting from \verb+biber+'s
internal LaTeX character macro decoding break \verb+inputenc+. This is
because \verb+inputenc+ does not implement all of UTF-8, only a
commonly used subset.

An example--if you had \verb+\DJ+ in your \verb+.bib+ data source,
\verb+biber+ decodes this correctly to «Đ» and this breaks \verb+inputenc+
because it doesn't understand that UTF-8 character. The real solution here
is to switch to a TeX engine with full UTF-8 support like XeTeX or LuaTeX
as these don't use or need \verb+inputenc+. However, you can also try the
\verb+--bblsafechars+ option which will try to convert any UTF-8 chars into
LaTeX macros on output. The \verb+biblatex+ option
«\verb+texencoding=ascii+» (which corresponds to the \verb+biber+ option
«\verb+--bblencoding|-E+») will automatically set \verb+--bblsafechars+.

See also the \verb+biber --help+ output for the
\verb+--bblsafecharsset+ option which can customise the set of
conversion characters to use.
\bigskip
\minisec{Decoding for internal sorting}

If your \verb+bblencoding+ is not UTF-8, and you are using some UTF-8
equivalent LaTeX character macros in your \verb+.bib+ data source, then some
\verb+.bbl+ fields (currently only \verb+\sortinit{}+) might end up
with invalid characters in them, according to the \verb+.bbl+
encoding. This is because some fields must be generated from the final
sorting data which is only available after the LaTeX character macro
decoding step.

For example, suppose you are using PDFLaTeX with\\
\verb+\usepackage[latin1]{inputenc}+ and the following \verb+bibtex+
data source entry:

\begin{verbatim}
@BOOK{citekey1,
  AUTHOR = {{\v S}imple, Simon},
}
\end{verbatim}

\noindent With normal LaTeX character macro decoding, the
\verb+{\v S}+ is decoded into «Š» and so with name-first sorting,
\verb+\sortinit{}+ would be «Š». This is an invalid character in
latin1 encoding and so the \verb+.bbl+ would be broken. In such cases
when \verb+\sortinit{}+ is a char not valid in the \verb+bblencoding+,
\verb+biber+ tries to replace the character with a suitable LaTeX
macro. The solution is really to use UTF-8 \verb+.bbl+ encoding whenever
possible. In extreme cases where even with UTF-8 encoding,
the char is not recognised by LaTeX due to an incomplete UTF-8
implementation (as with \verb+inputenc+), this might also mean
switching TeX engines to one that supports full UTF-8.

\subsubsection{Examples}

\biberex{biber}

\noindent Set \verb+bibencoding+ and \verb+bblencoding+ from the
config file or \verb+.bcf+.

\biberex{biber --bblencoding=latin2}

\noindent Encode the \verb+.bbl+ as latin2, overriding the
\verb+.bcf+.

\biberex{biber --bblsafechars}

\noindent Set \verb+bibencoding+ and \verb+bblencoding+ from the
config file or \verb+.bcf+. Force encoding of UTF-8 chars to LaTeX
macros using default conversion set.

\biberex{biber --bblencoding=ascii}

\noindent Encode the \verb+.bbl+ as ascii, overriding the
\verb+.bcf+. Automatically sets \verb+--bblsafechars+ to force UTF-8
to LaTeX macro conversion.

\biberex{biber --bblencoding=ascii --bblsafecharsset=full}

\noindent Encode the \verb+.bbl+ as ascii, overriding the
\verb+.bcf+. Automatically sets \verb+--bblsafechars+ to force UTF-8
to LaTeX macro conversion using the full set of conversions

\biberex{biber --decodecharsset=full}

\noindent Set \verb+bibencoding+ and \verb+bblencoding+ from the
config file or \verb+.bcf+. Use the full LaTeX macro to UTF-8
conversion set because you have some more obscure character macros in
your \verb+.bib+ data source which you want to sort correctly

\biberex{biber -u}

\noindent Shortcut alias for \verb+biber --bibencoding=UTF-8+

\biberex{biber -U}

\noindent Shortcut alias for \verb+biber --bblencoding=UTF-8+

\subsection{Editor Integration}

Here is some information on how to integrate \verb+biber+ into some of the
more common editors

\subsubsection{Emacs}

Emacs has the very powerful AUcTeX mode for editing TeX and running
compilations. BibTeX is already integrated into AUCTeX and it is quite
simple to add support for \verb+biber+. Use the Emacs Customise interface to
modify the \verb+TeX-command-list+ variable and add a \verb+Biber+ command.

\begin{verbatim}
M-x customise-variable
TeX-command-list
\end{verbatim}

\noindent and then \verb+Ins+ somewhere a new command that looks like
Figure \ref{fig:biber-auctex}.

\begin{figure}[!htbp]
  \centering
  \includegraphics[width=4in,keepaspectratio=true]{biber-auctex.png}
  \caption{Screenshot of AUCTeX command setup for Biber}
  \label{fig:biber-auctex}
\end{figure}

\noindent Alternatively, you can add it directly in lisp to your \verb+.emacs+ like
this:

\begin{verbatim}
(add-to-list 'TeX-command-list
  (quote
    ("Biber" "biber %s" TeX-run-Biber nil t :help "Run Biber")))
\end{verbatim}

\noindent However you add the command to \verb+TeX-command-list+, customise the
actual \verb+Biber+ command parameters as you want them, using «\verb+%s+» as
the LaTeX file name place holder. Then define the following two functions
in your \verb+.emacs+.

\small
\begin{verbatim}
(eval-after-load "tex"
  (quote (defun TeX-run-Biber (name command file)
    "Create a process for NAME using COMMAND to format FILE with Biber." 
    (let ((process (TeX-run-command name command file)))
      (setq TeX-sentinel-function 'TeX-Biber-sentinel)
      (if TeX-process-asynchronous
          process
        (TeX-synchronous-sentinel name file process))))
         )
  )

(eval-after-load "tex"
  (quote (defun TeX-Biber-sentinel (process name)
    "Cleanup TeX output buffer after running Biber."
    (goto-char (point-max))
    (cond
     ;; Check whether Biber reports any warnings or errors.
     ((re-search-backward (concat
                           "^(There \\(?:was\\|were\\) \\([0-9]+\\) "
                           "\\(warnings?\\|error messages?\\))") nil t)
      ;; Tell the user their number so that she sees whether the
      ;; situation is getting better or worse.
      (message (concat "Biber finished with %s %s. "
                       "Type `%s' to display output.")
               (match-string 1) (match-string 2)
               (substitute-command-keys
                "\\<TeX-mode-map>\\[TeX-recenter-output-buffer]")))
     (t
      (message (concat "Biber finished successfully. "
                       "Run LaTeX again to get citations right."))))
    (setq TeX-command-next TeX-command-default))
         )
  )
\end{verbatim}
\normalsize

\noindent You'll then see a \verb+Biber+ option in your AUCTeX command menu or
you can just \verb+C-c C-c+ and type \verb+Biber+.

\subsubsection{TeXworks}

It's very easy to add \verb+biber+ support to TeXworks. In the Preferences,
select the Typesetting tab and then add a new Processing Tool as in Figure
\ref{fig:biber-texworks}.

\begin{figure}[!htbp]
  \centering
  \includegraphics[width=4in,keepaspectratio=true]{biber-texworks.png}
  \caption{Screenshot of TeXworks processing tool setup for Biber}
  \label{fig:biber-texworks}
\end{figure}

\subsection{BibTeX macros and the MONTH field}

\verb+BibTeX+ defines macros for month abbreviations
like «\verb+jan+», «\verb+feb+» etc. \verb+biber+ also does this,
defining them as numbers since that is what \verb+biblatex+ wants. In
case you are also defining these yourself (although if you are only
using \verb+biblatex+, there isn't much point), you will get macro
redefinition warnings from the \verb+btparse+ library. You can turn
off \verb+biber+'s macro definitions to avoid this by using the option
\verb+--nostdmacros+.

\verb+biber+ will look at any \verb+MONTH+ field in a \verb+bibtex+ data
source and if it's not a number as \verb+biblatex+ expects (because it
wasn't one of the macros as mentioned above or these macros were disabled
by \verb+--nostdmacros+), it will try to turn it into the right number in
the \verb+.bbl+. If you only use \verb+biblatex+ with your \verb+bibtex+
data source files, you should probably make any \verb+MONTH+ fields be the
numbers which \verb+biblatex+ expects.

\subsection{Biber data source drivers}\label{dcf}

\verb+biber+ uses a modular data source driver model to provide access
to supported data sources. The drivers are responsible for mapping
driver entry types and fields to the \verb+biblatex+ model. This is
aided by a driver configuration file (\verb+.dcf+). The information in
this file for each driver can be found in the driver documentation folder on
SourceForge\footnote{\url{https://sourceforge.net/projects/biblatex-biber/files/biblatex-biber/current/documentation/drivers}}. This
file shows you which handlers the driver uses for different fields and
what certain entry types and fields are aliased to in the
\verb+biblatex+ data model. This is not fantastically useful to know
without knowing also what the named driver handlers do specifically but there
is a «description» section which mentions any special handling and
general comments on the driver. You should read the documentation for
the drivers you use to get an idea of how \verb+biber+ handles your
data. Data model mapping is an imprecise art
and the drivers are where the necessarily messy parts of \verb+biber+
live. Most data source models are not designed with typesetting in
mind and are usually not fine-grained enough to provide the sorts of
information that \verb+biblatex+ needs. \verb+biber+ does its best to
obtain as much meaningful information from a data source as possible.
Currently supported data sources drivers are:

\begin{itemize}
\item \verb+bibtex+ --- BibTeX data files
\item \verb+endnotexml+ --- Endnote XML export format, version $\geq$ Endnote X1
\item \verb+ris+ --- Research Information Systems format
\item \verb+zoterordfxml+ --- Zotero RDF XML format, version 2.0.9
\end{itemize}

\section{Binaries}\label{binaries}

\verb+biber+ is a Perl application which relies heavily on quite a few
modules. It is packaged as a stand-alone binary using the excellent
\verb+PAR::Packer+ module which can pack an entire \verb+Perl+ tree plus
dependencies into one file which acts as a stand-alone binary and is
indistinguishable from such to the end user. You can also simply download
the \verb+Perl+ source and run it as a normal \verb+Perl+ program which
requires you to have a working \verb+Perl+ 5.10+ installation and the
ability to install the pre-requisite modules. You would typically only do
this if you wanted to keep up with all the bleeding-edge git commits before
they had been packaged as a binary. Almost all users will not want to do
this and should use the binaries from their TeX distribution or downloaded
directly from SourceForge in case they need to use a more recent binary
than is included in their TeX distribution.

The binary distributions of biber are made using the \verb+Perl+ \verb+PAR::Packer+
module. They can be used as a normal binary but have some behaviour which
is worth noting:

\begin{itemize}
\item Don't be worried by the size of the binaries. \verb+PAR::Packer+ essentially
  constructs a self-extracting archive which unpacks the needed files first.
\item On the first run of a new version (that is, with a specific hash),
  they actually unpack themselves to a temporary location which varies by
  operating system. This unpacking can take a little while and only happens
  on the first run of a new version. \textbf{Please don't kill the process
    if it seems to take some time to do anything on the first run of a new
    binary}. If you do, it will not unpack everything and it will almost
  certainly break \verb+biber+. You will then have to delete your binary
  cache (see section \ref{bc} below) and re-run the \verb+biber+ executable
  again for the first time to allow it to unpack properly.
\end{itemize}

\subsection{Binary Caches}\label{bc}

\verb+PAR::Packer+ works by unpacking the required files to a cache
location. It only does this on the first run of a binary 
by computing a hash of the binary and comparing it with
the cache directory name which contains the hash. So, if you run
several versions of a binary, you will end up with several cached
trees which are never used. This is particularly true if you are regularly
testing new versions of the \verb+biber+ binary. It is a good idea to
delete the caches for older binaries as they are not needed and can take up
a fair bit of space. The caches are located in a temporary location which
varies from OS to OS. The cache name is:\\[1ex]

\noindent\verb+par-<username>/cache-<hash>+ (Linux/Unix/OSX)\\
\verb+par-<username>\cache-<hash>+ (Windows)\\[1ex]

\noindent The temp location is not always obvious but these are sensible
places to look (where \verb+*+ can vary depending on username):

\begin{itemize}
\item \verb+/var/folders/*/*/*/+ (OSX, local GUI login shell)
\item \verb+/var/tmp/+ (OSX (remote ssh login shell), Unix)
\item \verb+/tmp/+ (Linux)
\item \verb+C:\Documents and Settings\<username>\Local Settings\Temp+ (Windows/Cygwin)
\item \verb+C:\Windows\Temp+ (Windows)
\end{itemize}

\noindent To clean up, you can just remove the whole \verb+par-<username>+
directory/folder and then run the current binary again.

\subsection{Binary Architectures}

Binaries are available for many architectures, directly on SourceForge and
also via TeXLive:

\begin{itemize}
\item \verb+linux_x86_32+
\item \verb+linux_x86_64+
\item \verb+MSWin32+
\item \verb+cygwin32+
\item \verb+darwin_x86_64+
\item \verb+darwin_x86_i386+
\item \verb+freebsd_x86+\tpb
\item \verb+freebsd_amd64+\tpb
\item \verb+solaris_x86+\tpb
\end{itemize}

\noindent If you want to run development versions, they are usually only
regularly updated for the core architectures which are not flagged as
third-party built above. If you want to regularly run the latest
development version, you should probably git clone the relevant branch and
run \verb+biber+ as a pure perl program directly.

\subsection{Installing}

These instructions only apply to manually downloaded binaries. If
\verb+biber+ came with your TeX distribution just use it as normal.

Download the binary appropriate to you
OS/arch\footnote{\url{https://sourceforge.net/projects/biblatex-biber}}. Below
I assume it's on your desktop.

You have to move the binary to somewhere in you command-line or TeX utility
path so that it can be found. If you know how to do this, just ignore the
rest of this section which contains some instructions for users who are
not sure about this.

\subsubsection{OSX}

If you are using the TexLive MacTeX distribution:

\begin{verbatim}
sudo mv ~/Desktop/biber /usr/texbin/
sudo chmod +x /usr/texbin/biber
\end{verbatim}

\noindent If you are using the macports TexLive distribution:

\begin{verbatim}
sudo mv ~/Desktop/biber /opt/local/bin/
sudo chmod +x /opt/local/bin/biber
\end{verbatim}

\noindent The «\verb+sudo+» commands will prompt you for your password.

\subsubsection{Windows}

The easiest way is to just move the executable into your \verb+C:\Windows+ directory since
that is always in your path. A more elegant is to put it somewhere in
your TeX distribution that is already in your path. For example if you
are using MiKTeX:

\begin{verbatim}
C:\Program Files\MiKTeX 2.8\miktex\bin\
\end{verbatim}

\subsubsection{Unix/Linux}

\begin{verbatim}
sudo mv ~/Desktop/biber /usr/local/bin/biber
sudo chmod +x /usr/local/bin/biber
\end{verbatim}

\noindent Make sure \verb+/usr/local/bin+ is in your PATH. Search Google for «set PATH
linux» if unsure about this. There are many pages about this, for example:
\url{http://www.cyberciti.biz/faq/unix-linux-adding-path/}


\subsection{Building}

Instructions for those who want/need to build an executable from the
\verb+Perl+ version. For this, you will need to have \verb+Perl+ 5.10+ with
the following modules:

\begin{itemize}
\item All \verb+biber+ pre-requisites
\item \verb+PAR::Packer+ and all dependencies
\end{itemize}

\noindent You should have the latest CPAN versions of all required modules
as \verb+biber+ is very specific in some cases about module versions and
depends on recent fixes in many cases. You can see if you have the
\verb+biber+ \verb+Perl+ dependencies by the usual

\begin{verbatim}
perl ./Build.PL
\end{verbatim}

\noindent invocation in the \verb+biber+ \verb+Perl+ distribution tree
directory. Normally, the build procedure for the binaries is as
follows\footnote{On UNIXequse systems, you may need to specify a full
  path to the scripts e.g. \texttt{./Build}}:

\begin{itemize}
\item Get the biber source tree from SF and put it on the architecture
  you are building for
\item cd to the root of the source tree
\item \verb+perl Build.PL+ (this will check your module
  dependencies)
\item \verb+Build test+
\item \verb+Build install+ (may need to run this as sudo on
  UNIXesque systems)
\item \verb+cd dist/<arch>+
\item \verb+build.sh+ (\verb+build.bat+ on Windows)
\end{itemize}

\noindent This leaves a binary called «\verb+biber-<arch>+» (also with
a «\verb+.exe+» extension on Windows/Cygwin) in your current directory.
The tricky part is constructing the information for the build
script. There are two things that need to be configured, both of
which are required by the \verb+PAR::Packer+ module:

\begin{enumerate}
\item A list of modules/libraries to include in the binary which are not
  automatically detected by the \verb+PAR::Packer+ dependency
  scanner
\item A list of extra files to include in the binary which are not
  automatically detected by the \verb+PAR::Packer+ dependency
  scanner
\end{enumerate}

\noindent To build \verb+biber+ for a new architecture you need to
define these two things as part of constructing new build scripts:

\begin{itemize}
\item Make a new subfolder in the \verb+dist+ directory named after the
  architecture you are building for. This name is arbitrary but should
  be fairly obvious like «\verb+solaris-sparc-64+», for example.
\item Copy the \verb+biber.files+ file from an existing build
  architecture into this directory.
\item For all of the files with absolute pathnames in there (that is,
  ones we are not pulling from the \verb+biber+ tree itself), locate these
  files in your \verb+Perl+ installation tree and put the correct path in the
  file.
\item Copy the build script from a vaguely similar architecture
  (i.e. Windows/non-Windows \ldots) to your new architecture
  directory. 
\item Change the \verb+--link+ options to point to where the required
  libraries reside on your system.
\item Change the \verb+--output+ option to name the resulting binary
  for your architecture.
\item Run the build script
\end{itemize}

\noindent The \verb+--link+ options can be a little tricky
sometimes. It is usually best to build without them once and then run
\verb+ldd+ (or OS equivalent) on the binary to see which
version/location of a library you should link to. You can also try
just running the binary and it should complain about missing libraries
and where it expected to find them. Put this path into the
\verb+--link+ option. The \verb+--module+ options are the same for all
architectures and do not need to be modified.
On architectures which have or can have case-insensitive file systems,
you should use the build script from either Windows or OSX as a reference
as these include a step to copy the main \verb+biber+ script to a new name
before packing the binary. This is required as otherwise a spurious
error is reported to the user on first run of the binary due to a name
collision when it unpacks itself.

See the \verb+PAR+ wiki
page\footnote{\url{http://par.perl.org/wiki/Main_Page}} for FAQs and help
on building with \verb+PAR::Packer+. Take special note of the FAQs on
including libraries with the packed
binary\footnote{\url{http://par.perl.org/wiki/FAQ}, section entitled «My
  PAR executable needs some dynamic libraries»}.

\subsubsection{Testing a binary build}
You can test a binary that you have created by copying it to a machine
which preferably doesn't have \verb+perl+ at all on it. Running the binary with no
arguments will unpack it in the background and display the help. To really
test it without having LaTeX available, get the two quick test files from
SourceForge\footnote{\url{https://sourceforge.net/projects/biblatex-biber/files/biblatex-biber/testfiles}},
put them in a directory and run \verb+biber+ in that directory like this:

\begin{verbatim}
biber --validate_control --convert_control test
\end{verbatim}

\noindent This will run \verb+biber+ normally on the test files plus it
will also perform an XSLT transform on the \verb+.bcf+ and
leave an HTML representation of it in the same directory thus testing the
links to the XML and XSLT libraries as well as the \verb+bibtex+ parsing
libraries. The output should look something like this (may be differences
of \verb+biber+ version and locale of course but there should be no errors
or warnings).

\begin{verbatim}
INFO - This is biber 0.9.5
INFO - Logfile is 'test.blg'
INFO - BibLaTeX control file 'test.bcf' validates
INFO - Converted BibLaTeX control file 'test.bcf' to 'test.bcf.html'
INFO - Reading 'test.bcf'
INFO - Found 1 citekeys in bib section 0
INFO - Processing bib section 0
INFO - Looking for bibtex format file 'test.bib' for section 0
INFO - Found bibtex data file 'test.bib'
INFO - Decoding LaTeX character macros into UTF-8
INFO - Sorting list 'MAIN' keys
INFO - No sort tailoring available for locale 'en_GB.UTF-8'
INFO - Sorting list 'SHORTHANDS' keys
INFO - No sort tailoring available for locale 'en_GB.UTF-8'
INFO - Writing 'test.bbl' with encoding 'UTF-8'
INFO - Output to test.bbl
\end{verbatim}

\noindent There should now be these new files in the directory:

\begin{verbatim}
test.bcf.html
test.blg
test.bbl
\end{verbatim}


\end{document}
